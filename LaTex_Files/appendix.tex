\section{Evanescence in waveguides}
In all optical fibers, light propagates by means of total internal reflection, wherein the propagating light is launched into waveguide at angles such that upon reaching the cladding-core interface, the energy is reflected and remains in the core of the fiber. Remarkably, however, for light reflecting at angles near the critical angle, a significant portion of the power extends into the cladding or medium which surrounds the core. This phenomenon, known as the evanescent wave, extends only to a short distance from the interface, with power dropping exponentially with distance. A detailed discussion can be found in~\cite{Anderson2008}.

\section{Finite Element Method}
The finite element method (FEM) is a numerical technique for solving problems which are described by partial differential equations or can be formulated as functional minimization. A domain of interest is represented as an assembly of finite elements. Approximating functions in finite elements are determined in terms of nodal values of a physical field which is sought. A continuous physical problem is transformed into a discretized finite element problem with unknown nodal values. For a linear problem a system of linear algebraic equations should be solved. Values inside finite elements can be recovered using nodal values. For a detailed discussion on FEM, visit \href{https://www.comsol.co.in/multiphysics/finite-element-method}{COMSOL} official website.



