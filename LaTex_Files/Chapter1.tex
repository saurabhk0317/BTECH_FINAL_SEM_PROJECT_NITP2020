%---------------------------CHAPTER - 01----------------------------
%\linespread{1.5} 
\normalsize
In recent years, increasing demand of high data-rate devices and ultra-compact integrated circuits have pulled researcher's interest towards optical fiber communication. It is observed that state of the art optical devices tends to have bandwidth in terahertz region and can operate at very high frequencies. As the demand of high data-rate devices is increasing significantly in day-to-day life, world is becoming more reliant on optical communication than the electronics communication system. Many signal processing devices are being replaced by optical devices. Usually optical devices tend to have bandwidth in terahertz range.

Plasmonic waveguides and waveguide-based components are considered as the ideal candidates for highly integrated nanoscale photonic electronic circuit applications~\cite{Barnes2003,Ozbay2006}. Different kinds of plasmonic waveguiding structures have been proposed and developed in recent years, including metallic strips and nanowires, coupled nanowires, metallic grooves, metal-insulator-metal (MIM) waveguides ~\cite{Feng2007}, dielectric-loaded plasmonic waveguides ~\cite{Reinhardt2009,Hong2009}, as well as hybrid plasmonic waveguides ~\cite{Wang2019,Nikoufard2017,Hong2010,Oulton2008}.

\section{Plasmonic Waveguides}
The conventional dielectric waveguides which works on the principle of total internal reflection at the core/cladding interface, tend to suffer from diffraction limit. As a consequence, the minimum core size of conventional dielectric waveguides is of the order of the wavelength of light, posing a fundamental limit to the size scaling of integrated optical chips~\cite{Feng2007}. However, several schemes have been already proposed in order to transport optical information over long distances by means of deep subwavelength modes~\cite{Ohtsu2002,Takahara2004}.

Recently, dielectric slot-waveguide structures have been proposed, able to confine light fields over nanoscale low-refractive-index dielectric regions~\cite{Almeida2004,Feng2006,Xu2004}. These structures are based on nanoscale regions sandwiched by high-refractive-index dielectrics. It has been shown that, depending on the actual design characteristics, deep subwavelength localization and sizeable field-enhancement effects can be achieved due to the large-refractive-index discontinuity at the dielectric boundary regions. An alternative approach to achieve high field confinement is based on the excitation of surface plasmon polaritons (SPPs) in metal/dielectric waveguide structures. SPPs are surface localized light waves at dielectric–metal interfaces which are coupled to free electron oscillations in the metal.

\section{Hybrid Plasmonic Waveguides}
The hybrid plasmonic waveguides have drawn substantial research interest because of the exciting potential for integrating metallic and semiconductor nanostructures to fully exploit the advantages of both metals (light concentration and high electrical conductivity) and semiconductors (light emission and photocurrent generation). Moreover, the presence of a high refractive index semiconductor, separated from the metal film by a low refractive index layer, yields the best trade-off between the propagation loss and the power confinement ~\cite{Oulton2008}. Several hybrid plasmonic waveguide structures have been proposed and discussed both in theory and experiment; they include hybrid dielectric-loaded and doubled-hybrid plasmonic waveguide~\cite{Oulton2008,Hong2010}. Among these kinds of waveguide, the hybrid dielectric-loaded plasmonic waveguide (HDPW) has received particular attention as a transmission medium for various plasmonic components of high optical performance, such as waveguide-bends and couplers, as well as lasers at a deep sub-wavelength scale. Moreover, the HDPW structures have no minimum thickness limitation of the low refractive index layer, and thus achieve higher confinement while minimizing the scattering losses by virtue of their smoother layer interfaces.

Over past few years, HPW has been used for variety of applications. Our study here is restricted only to optical ring resonators (ORR) and optical nanoantennas.

\subsection{Optical Ring Resonator}
Ring resonators play an important role in the success of silicon photonics, because silicon enables ring resonators of an unprecedented small size. A generic ring resonator consists of an optical waveguide which is looped back on itself, such that a resonance occurs when the optical path length of the resonator is exactly a whole number of wavelengths. Ring resonators therefore support multiple resonances, and the spacing between these resonances, the free spectral range (FSR), depends on the resonator length. As discussed in~\cite{Bogaerts2011}, for many applications it is preferred to have a relatively large FSR (several nm), and this implies the use of small rings. This translates into a very hard requirement for the optical waveguide: to make a compact ring, a small bend radius is required, and this in turn is only possible with high-contrast waveguides with strong confinement.

A ring resonator as a stand-alone device only becomes useful when there is a coupling to the outside world. The most common coupling mechanism is using codirectional evanescent coupling between the ring and an adjacent bus waveguide. As will be discussed in more detail in the next section, the transmission spectrum of the bus waveguide with a single ring resonator will show dips around the ring resonances. This way, the ring resonator behaves as a spectral filter, which can be used for applications in optical communication, especially wavelength division multiplexing (WDM).

In this project, we have studied the characteristics of optical ring resonator (ORR) based on two different types of HPW structures; hybrid dielectric-loaded plasmonic waveguides (HDLW)~\cite{Hong2011} and hybrid metal-insulator-metal (HMIM) plasmonic waveguide~\cite{PSharma2016}. The HDLW based resonator structure consists of SOI-configuration with silica (SiO$_2$) as low refractive index layer and silver (Ag) as metal layer. While HMIM waveguide-based resonator consists of Aluminum Gallium Arsenide (AlGaAs) and poly tetra fluoro ethylene (PTFE) as high and low dielectric layer respectively.

\subsection{Optical Nanoantenna}
Optical nanoantenna efficiently convert free-propagating optical radiation into localized energy and vice versa. Therefore, the potential applications of optical antenna are closely related to their ability to strongly localized optical fields likewise; scattering, sensing inter- and/or intra- chip communications. Moreover, optical nanoantenna can play a vital role in reducing power consumption and allow higher speed for on-chip optical interconnects. As a result, it is considered as a prominent alternative to optical waveguide-based interconnects for on-chip wireless communication with high spectral efficiency.

Designing of a nanoantenna for wireless optical communication involves the basic principle: optimization of shape in order to match the impedance. It also takes into account for the parameters of antenna likewise; gain, directivity and reflection-coefficient.

In this context, we have studied the characteristics of horn-nanoantenna with silicon-on-insulator (SOI) configuration ~\cite{Nourmohammadi2019,Nikoufard2018}. It is able to operate in a range of optical frequencies by using a SOI-based horn-shaped nanoantenna. This nanoantenna is broadband at a wider range of frequencies (160–400 THz) in comparison with the previous antennas. This nanoantenna is capable of radiating all three optical communication wavelengths of 1550 nm (193.5 THz), 1310 nm (229 THz) and 850 nm (352.9 THz) with slightly higher gains at all wavelengths and smaller dimensions. All these features make this nanoantenna a good candidate for energy-harvesting applications or integration with devices such as dual laser combiner modules in communication systems. The horn nanoantenna has also been explored for several applications in near-infrared, far-infrared and visible region.