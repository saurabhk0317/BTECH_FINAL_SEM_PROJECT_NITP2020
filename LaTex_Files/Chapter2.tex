%\linespread{1.5} 
\normalsize
\section{Literature Review on HPW}
In recent years, hybrid plasmonic waveguides have been a demanding area of research in photonics. One of the main reasons for the increasing demand of the hybrid plasmonic waveguide is that its ability to confine light at nanoscale with long range propagation. In past decade, researchers have studied HPW for number of applications. Some of them are documented in the literature~\cite{Oulton2008,Hong2011,Alam2015,Tsilipakos2016,PSharma2016,Nikoufard2017,Nikoufard2018,Nourmohammadi2019,Wang2019,PKumar2020}. These authors have utilized different properties of hybrid plasmonic waveguides for various applications.

\cite{Oulton2008} proposed a dielectric nanowire-based structure to achieve long range propagation with subwavelength confinement. In this structure, a dielectric nanowire is separated from a metal surface by a nanoscale dielectric gap such that the 'capacitor-like' energy storage is enabled by the coupling between the plasmonic and waveguide modes across the gap.

\cite{Hong2011} analyzed a hybrid dielectric-loaded plasmonic waveguide for long range propagation and strong field confinement. They explored this structure to study the characteristics of various wavelength selective components for high optical performance.

\cite{Alam2015} utilized the properties of the mode supported by hybrid plasmonic waveguides. Considering silicon compatibility of the HPW, a number of hybrid waveguide devices were presented for silicon on insulator (SOI) platform.

\cite{Tsilipakos2016} proposed a disk resonator based on hybrid plasmonic waveguide for Kerr bistability and self-pulsation, operating with mW input powers. They attempted to reduce the input power requirement for various devices in silicon photonics.

\cite{PSharma2016} analyzed the mode properties of a two layered HP-waveguide based on metal-insulator-metal (MIM) configuration and compared the results with that of single layered HP-waveguide. To understand the mode confinement of both waveguides, mode effective areas of both structures were compared. Further, a ring resonator based on hybrid metal-insulator-metal (HMIM) waveguide was simulated and its transmission characteristics were studied.

\cite{Nikoufard2017} studied the mode properties of InP-based deeply etched HP-waveguide and compared the results with that of InP-based conventional waveguide. Utilizing various properties of InP-based deeply etched HP-waveguide, a nanoscale multimode interference (MMI) power splitter was proposed. The effect of various MMI parameters such as length, width, and wavelength on optical power transmission was investigated. Further, ~\cite{Nikoufard2018} utilized InP-based HP-waveguide to propose a bowtie nanoantenna with the capability of monolithic integration with laser and photodetector.

\cite{Nourmohammadi2019} proposed a horn-shaped nanoantenna for broadband nanophotonic applications. This structure utilized a HP-waveguide with SOI-configuration. It was observed that proposed structure is able to operate in a wide optical frequency range of 160--400 THz and includes optical communication wavelengths of 850, 1310 and 1550 nm.

\cite{Wang2019} proposed a silicon-based HP-waveguide with the goal of obtaining long range propagation at optical communication wavelength of $\lambda$ = 1550 nm. This structure is designed with SOI-rib and a metal cap at the top. Some potential applications for the proposed structure were investigated.

\cite{PKumar2020} investigated a multi-layered hybrid metal-insulator-metal plasmonic waveguide which was observed to have a relatively smaller mode effective area in the low-dielectric region at optical communication wavelength of 1550 nm. Adding further to that, proposed structure was explored for crosstalk estimation.

Although most of the authors in their recent work shown that hybrid plasmonic waveguide poses tremendous qualities in terms of subwavelength confinement, few authors have pointed out that it tends to suffer from large loss. So, there is a trade-off between confinement and loss which limits the capability of hybrid plasmonic waveguides at some extent.