\thispagestyle{plain}
\linespread{1.5}
\begin{center}
	\fontsize{14}{18}\selectfont \textbf{ABSTRACT}
%	\textbf{\textbf{\fontsize{14pt}{24pt}\selectfont ABSTRACT}}
\end{center}

\vspace{0.3cm}

% Start typing out your abstract after the \selectfont tag. And after your last line add a \\ if you require more spacing

%\fontsize{12pt}{18pt}\selectfont 
\normalsize
\noindent
This project aims to the study of hybrid plasmonic waveguides (HPWs) and their potential applications in the field of photonics. Here, we have presented our effort toward recent advancement in the photonic integrated circuits and related studies. We have simulated various types of HPW structures, proposed over past few years and studied their characteristics such as effective mode index, mode effective area, and propagation length. It is observed that there is a trade-off between mode confinement and propagation loss in case of most of the HP-waveguides. Structures with smaller mode effective area tend to have larger propagation loss and vice-versa.

Further, we have investigated a few potential applications of HP-waveguides such as optical ring resonators (ORR) and optical nanoantenna. We have simulated a hybrid dielectric-loaded plasmonic waveguide (HDLW) based ring resonator and studied its transmission characteristics at both constructive and destructive interference wavelengths. A high extinction ratio of 22.5 dB and 24.9 dB is obtained at r = 2 $\upmu \text{m}$ and 5 $\upmu \text{m}$ respectively. While low transmission loss of 0.8 dB and 0.5 dB is obtained at radii 2 $\upmu \text{m}$ \& 5 $\upmu \text{m}$ respectively. For nanoantenna, we have studied the characteristics of a HP-waveguide horn-nanoantenna and simulated its input HP-waveguide which is based on silicon-on-insulator (SOI) configuration. All simulations have been carried out using finite element method (FEM) in the COMSOL Multiphysics simulation software and graphs are plotted with the help of MATLAB.


\newpage
